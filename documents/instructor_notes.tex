\documentclass[11pt]{article}
\usepackage[margin=1.0in]{geometry}

% Required packages
\usepackage{amsmath,amsfonts,amsthm,amssymb} % Mathematical typesetting and symbols
\usepackage{bigstrut} % Row spacing
\usepackage{enumerate} % Custom enumerate labels
\usepackage{esint} % Alternate double integral symbols
\usepackage{fancyhdr} % Headers and footers
\usepackage{graphicx} % Figure inclusion
\usepackage{hyperref} % Hyperlinks for citations, references, and URLs
\usepackage{placeins} % Float barriers
\usepackage{slashbox} % Slash in a table cell
\usepackage{subcaption} % Captions for subfigures
\usepackage{titlesec} % Custom section labels
\usepackage{wrapfig} % Wrapping text around figures
\usepackage{xcolor} % Text color

% Label Sections as Parts
\titleformat{\section}{\normalfont\Large\bfseries}{Part \thesection:}{0.5em}{}

% Define license image
\newcommand{\licenseimage}{figures/by-sa.png}

% Footer for first page
\pagestyle{plain}
\renewcommand{\headrulewidth}{0pt}
\fancyhead{}
\fancyfoot{}
\cfoot{\includegraphics[width=0.75in]{\licenseimage} \\ \footnotesize This work by the \href{https://floridapoly.edu/academic-departments/applied-mathematics.php}{Department of Applied Mathematics at Florida Polytechnic University} is licensed under a \href{https://creativecommons.org/licenses/by-sa/4.0/}{Creative Commons Attribution-ShareAlike 4.0 International License}.}


%==============================================================================

\begin{document}

\thispagestyle{fancy} % Add footer to first page only

This computer lab is meant to be used as an introductory activity in a Calculus~I course. It asks the students to explore concepts like average and instantaneous rates of change from a computational perspective using Excel spreadsheets.

%==============================================================================
\subsection*{Learning Goals}

The primary mathematical goal of this activity is to introduce students to fundamental concepts of differential calculus by allowing them to experiment with computing rates of change, however there is also a soft goal of removing students from their comfort zones to try to get them to think about college-level mathematics in a new way. As such a large number of the activities revolve around the student figuring out how to apply their understanding of a concept to answer an open-ended question, like being able to look at a rate-of-change graph to answer qualitative questions about how a quantity changes at a particular point.

%==============================================================================
\subsection*{Prerequisites}

As this activity is meant for use at the beginning of Calculus~I, no knowledge of calculus is assumed, however it is assumed that students will have some familiarity with using Excel. In any case it may be a good idea to preface the lab with a brief Excel primer. Important functionality to point out includes:~using cell formulas, using relative and absolute cell references, and plotting.

%==============================================================================
\subsection*{Activity Notes}

The exact format for how this activity should be presented and what should be submitted is up to the instructor. The lab, itself, is relatively light on explanation and mostly consists of a sequence of numerical experiments for the students to go through and questions for them to answer. It is expected that the activity will be used alongside some instruction.

Throughout this activity, be sure to focus on the general approach that professional mathematicians often take to solving problems, which is quite different than what most high school students will be used to. Most students are comfortable memorizing and applying formulas, and have (understandably) come to believe that that's all math is. However the work that real mathematicians do places much more emphasis on conceptual understanding, problem solving, and numerical experimentation to come up with quick and dirty results that help to understand what's going on. Encourage the students to play around with ideas and to experiment with things, which the computer makes it easy to do quickly and easily.

Finally, the instructor should be aware of (and prepared to head off) a potentially dangerous line of thinking that may occur to some students after this activity, which goes:~``If we can approximate a rate of change using a difference quotient over a very small interval, why don't we just use that all the time? Why do we have to bother with limits or derivatives or any of the other methods or theory covered in this class?'' In fact, it might even be valuable to bring up this thought in class and to have students volunteer ideas for why it might be flawed. Some suggestions include:
\begin{itemize}
	\item It might seem counterintuitive at first, the more mathematically complex process of evaluating a limit of difference quotients actually tends to lead to simpler and more usable final results than finite difference quotients would. If we start with a relatively simple closed-form combination of elementary functions, then its derivative tends to also be a relatively simple closed-form combination of elementary functions.
	\item While it is true that a finite difference approximation provides a quick and easy (approximate) value of a derivative \textit{at one point}, it's less ideal when we're interested in finding values of the derivative at many points or at all points on a domain. Analytically evaluating a derivative by hand gives us a function, which allows us to instantly find the value of the derivative at any point we want, whereas the finite difference approximation needs to be recalculated from scratch every time we move to a new point. More importantly, the finite difference approximations can only provide a sequence of discrete (approximate) values for the derivative function, while finding a closed-form derivative by hand provides the entire continuous function.
	\item Derivatives appear in all sorts of applications for all sorts of reasons, and it's very rare that we \textit{just} want to compute one value of a derivative and that's that. One of the most important applications of derivatives is the differential equation, where instead of having a function and trying to find its derivative, we instead know a function's derivative and want to find what the function is. This is a problem that involves derivatives, but solving it requires a precise conceptual understanding of what a derivative is and how it's defined. If all we know how to do is numerically approximate derivatives, we can't solve differential equations.
\end{itemize}

%%%
%\subsubsection*{Part~1:~Instantaneous Velocity of a Weather Balloon}
\subsubsection*{Instantaneous Velocity of a Weather Balloon}

Many students fresh from high school or precalculus will be used to functions that look like
\begin{align*}
	f(x) = x^2 - 2x + 1
\end{align*}

i.e.~functions defined by simple closed-form formulas, and (again understandably) may believe that a function \textit{is} a simple closed-form formula. Spend some time highlighting the fact that a data table like the one included in the spreadsheet is still a perfectly good function, in the sense that it's still a rule that assigns an output to every input in a domain. Also highlight the fact that, in the real world, this is what most functions we deal with look like. Having a simple formula to work with is nice, but it usually only comes up as a simplified model of a far more complicated phenomenon, and in applied math we often have to deal with functions by using purely numerical computations. In spite of the fact that they'll be spending a lot of time in class learning how to analytically differentiate simple, closed-form functions, in practice we often \textit{have} to result to the types of numerical approximations that this activity uses. This also provides a great opportunity to highlight the importance of computers in modern mathematics, since having to compute thousands of cell values by hand would be virtually impossible.

This activity only obliquely references the fact that there is a relationship between slope and rate of change, in the sense that the student is expected to notice that the slope of the position graph corresponds at least qualitatively to the balloon's velocity. It is worth spending some class time highlighting the fact that the average velocity over an interval is actually \textit{exactly the same thing} as the slope of the secant line over that interval, easily seen since both are computed using the same difference quotient. %This discussion could also be saved for the next part.

Students are likely to struggle with the average velocity computations, which require a good working knowledge of how to combine relative and absolute cell references in Excel to keep the reference to time~60 constant while the other time increments. It would be useful to walk them through some demonstrations for how to use them.

Also be sure to show the students how to copy a formula into all nonempty rows by double-clicking on the lower right corner of the first cell. This is a very long data table, and clicking and dragging to copy formulas repeatedly will take a while.

The main purpose of the average velocity computations is to notice a couple of things about how average velocity varies as a function of the step size, so you might want to nudge the students in the right direction. In short, looking at the graph of the function, it seems like the average velocity is a nice, continuous, well-behaved function that just happens to have a hole at the one input we wanted to find (the one where the step size is zero). The fact that this graph appears continuous, but has a hole in it, is what motivates the need for limits in calculus, since limits are the mathematical tool we use to find which value ``should'' fill the hole.

This introductory problem uses velocity as an example of a rate of change. Be sure to emphasize the fact that ``rate of change'' could refer to anything changing with respect to anything else, not just position changing with respect to time.

\end{document}
