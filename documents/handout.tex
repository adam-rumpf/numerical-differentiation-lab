\documentclass[11pt]{article}
\usepackage[margin=1.0in]{geometry}

% Required packages
\usepackage{amsmath,amsfonts,amsthm,amssymb} % Mathematical typesetting and symbols
\usepackage{bigstrut} % Row spacing
\usepackage{enumerate} % Custom enumerate labels
\usepackage{esint} % Alternate double integral symbols
\usepackage{fancyhdr} % Headers and footers
\usepackage{graphicx} % Figure inclusion
\usepackage{hyperref} % Hyperlinks for citations, references, and URLs
\usepackage{placeins} % Float barriers
\usepackage{slashbox} % Slash in a table cell
\usepackage{subcaption} % Captions for subfigures
\usepackage{titlesec} % Custom section labels
\usepackage{wrapfig} % Wrapping text around figures
\usepackage{xcolor} % Text color

% Label Sections as Parts
\titleformat{\section}{\normalfont\Large\bfseries}{Part \thesection:}{0.5em}{}

% Define license image
\newcommand{\licenseimage}{figures/by-sa.png}

% Footer for first page
\pagestyle{plain}
\renewcommand{\headrulewidth}{0pt}
\fancyhead{}
\fancyfoot{}
\cfoot{\includegraphics[width=0.75in]{\licenseimage} \\ \footnotesize This work by the \href{https://floridapoly.edu/academic-departments/applied-mathematics.php}{Department of Applied Mathematics at Florida Polytechnic University} is licensed under a \href{https://creativecommons.org/licenses/by-sa/4.0/}{Creative Commons Attribution-ShareAlike 4.0 International License}.}


%==============================================================================

\begin{document}

\noindent \makebox[\textwidth]{\textbf{\LARGE Introduction to Rates of Change}}

\quad

%% Update cover image
\begin{figure}[h]
	\centering
	\includegraphics[height=0.4\textwidth]{figures/stables.jpg} % Stables (Stallungen), Franz Marc
\end{figure}

%==============================================================================
\section*{Introduction}

Calculus, broadly speaking, is the study of \textit{rates of change}. That is, calculus is the branch of mathematics required whenever we want to describe a how one quantity changes with respect to another. This makes it of particular importance in physics, which is largely about studying how objects move through space as time advances, but anything that changes over time (like a chemical reaction, an electrical charge, a population, or a stock price) can be studied more closely using calculus.

One of the most important rates of change in physics, and a good place for us to start our introduction to calculus, is \textit{velocity}, the rate of change in an object's position with respect to time. From high school physics you should already be familiar with one type of velocity:~the \textit{average velocity} over some interval of time. For example, if a car is driving along a highway and it crosses mile marker~74 at~2:16pm and then mile marker~106 at~2:41pm, then its average velocity between~2:16pm and~2:41pm is
\begin{align*}
	v_{\text{avg}} &= \frac{\text{change in position between 2:16 and 2:41}}{\text{change in time between 2:16 and 2:41}} = \frac{\text{106 mi $-$ 74 mi}}{\text{41 min $-$ 16 min}} = \frac{\text{32 mi}}{\text{25 min}} = 1.28~\frac{\text{mi}}{\text{min}}
\end{align*}

which is~$76.8$~mph. Average velocity is perfectly easy to define and compute (it's what's called a \textit{difference quotient}, which is exactly what it sounds like), but it doesn't tell us everything about the car's motion. The car's speed probably wasn't exactly~76.8~mph over the entire time interval; it was probably traveling slightly faster at some times and slightly slower at others. Presumably if we had more information (more positions at more times) we would be able to describe the car's \textit{instantaneous velocity}, or its velocity at a specific moment in time.

Instantaneous velocity seems like a simple concept, and in everyday speech we often talk about how fast something is happening ``right now'' or at some other specific moment in time. However, mathematics is all about making vague ideas precise enough to actually use in applications, and it turns out that the notion of instantaneous velocity is surprisingly tricky to define in a sensible, usable way. This worksheet will take you through a sequence of numerical experiments and activities to get you to play around with these ideas for yourself to build up an intuition about how to approach rates of change.

%%%
\subsection*{Required Materials}

This lab includes computational activities in the provided spreadsheet file \verb|rates_of_change.xlsx|. You will need Microsoft Excel\footnote{\url{https://www.microsoft.com/en-us/microsoft-365/excel}} installed in order to access and work with the file. If you cannot access Excel, there are several functionally-equivalent free alternatives that will work just as well (we recommend either Google Sheets\footnote{\url{https://www.google.com/sheets/}} or LibreOffice Calc\footnote{\url{https://www.libreoffice.org/discover/calc/}}).

It is expected that you have some basic familiarity with how to use Excel to perform basic computations, including how to use cell formulas, how to reference other cells, and how to generate plots. If not, the best way to learn is simply to try using it to solve some problems, asking for help from your classmates or the instructor or looking up help online when you get stuck. Using Excel is a skill, and like any skill there is no substitute for practice.

%%%
\subsection*{Learning Objectives}

After completing this project, you should be able to:
\begin{itemize}
	\item Explain how the idea of an instantaneous rate of change is related to average rates of change.
	\item Use a computer to help to answer questions about a function's rate of change, like when its value is increasing/decreasing/constant and when it is changing the fastest.
	\item Work with a real-world function defined as a table of values rather than as a simple formula.
	\item Be comfortable using a computer alongside analytical work by hand in order to solve a complicated problem.
\end{itemize}

%==============================================================================

% Instantaneous Velocity of a Weather Balloon
\newpage
\section{Instantaneous Velocity of a Weather Balloon}
\label{sec:balloon}

As noted above, average velocity is easy to compute, but we don't yet have a way to approach computing instantaneous velocities. In this activity you will work with a (lightly edited) set of position data gathered from a weather balloon launched from the Salton Sea weather station on February~28,~2021\footnote{F.~M.~Ralph, A.~M.~Wilson, R.~Demirdjian, D.~Alden, C.~Hecht, C.~J.~Ellis, B.~Kawzenuk, F.~Cannon, A.~Cooper, and K.~Paulsson. Radiosonde Data Collected During California Storms. UC~San~Diego Library Digital Collections, Dataset, 2021. \href{https://doi.org/10.6075/J09P31S0}{doi:10.6075/J09P31S0}} to try to see how we might approach the problem of how to describe the balloon's instantaneous velocity.

%%%
\subsection*{Activity}

\begin{enumerate}[\bf {\thesection}a.]
	\item Open the activity spreadsheet \verb|rates_of_change.xlsx| and make sure that you're on the first tab, labeled ``Weather Balloon''. You should see two columns of data:~column~\textbf{A} indicates the time~(in~seconds) since the balloon's launch and column~\textbf{B} indicates the position~(in~meters) of the balloon relative to its launch position\footnote{Actually, only the balloon's position in the longitudinal direction is included here. The original data set included latitudes and altitudes, but for this activity we're going to pretend that all of its motion occurs in a single direction. Describing more complicated movement in~3D space is a topic for Calculus~III.}.
\end{enumerate}









% Graph the position function, and answer a few initial questions by visual inspection (where does it look like it was moving the fastest, forward, backward, standing still, etc.)
% Lead up with some preliminary numerical experiments to find *average* velocities over a few specified intervals. Have the student observe how their estiamtes seem to behave as the interval shrinks. Include intervals that approach from both sides.
% Have the student use finite differences to plot velocity estimates over all possible intervals. Have them note what happens to the value exactly on top of the specified time. Have them graph only the nearby values to note the "hole".
% Since we only have discrete data there's a large gap in the rate of change graph, so the best we can do if we want a reasonable estimate of the instantaneous rate of change is to take one of the nearby values. Ask the student how the situation would be different if we had a closed-form function rather than a data table. Then there wouldn't be such a massive gap, but it would still be true that we can't use a step size of zero.
% Having established that we can't really use this method to find an average velocity between the same point twice, note that we can at least take a *small* interval, say between adjacent measurements. Generate an approximate rate-of-change function by using these single-step finite differences.
% Answer some questions by comparing the two graphs.




\end{document}
