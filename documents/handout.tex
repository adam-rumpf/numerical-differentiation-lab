\documentclass[11pt]{article}
\usepackage[margin=1.0in]{geometry}

% Required packages
\usepackage{amsmath,amsfonts,amsthm,amssymb} % Mathematical typesetting and symbols
\usepackage{bigstrut} % Row spacing
\usepackage{enumerate} % Custom enumerate labels
\usepackage{esint} % Alternate double integral symbols
\usepackage{fancyhdr} % Headers and footers
\usepackage{graphicx} % Figure inclusion
\usepackage{hyperref} % Hyperlinks for citations, references, and URLs
\usepackage{placeins} % Float barriers
\usepackage{slashbox} % Slash in a table cell
\usepackage{subcaption} % Captions for subfigures
\usepackage{titlesec} % Custom section labels
\usepackage{wrapfig} % Wrapping text around figures
\usepackage{xcolor} % Text color

% Label Sections as Parts
\titleformat{\section}{\normalfont\Large\bfseries}{Part \thesection:}{0.5em}{}

% Define license image
\newcommand{\licenseimage}{figures/by-sa.png}

% Footer for first page
\pagestyle{plain}
\renewcommand{\headrulewidth}{0pt}
\fancyhead{}
\fancyfoot{}
\cfoot{\includegraphics[width=0.75in]{\licenseimage} \\ \footnotesize This work by the \href{https://floridapoly.edu/academic-departments/applied-mathematics.php}{Department of Applied Mathematics at Florida Polytechnic University} is licensed under a \href{https://creativecommons.org/licenses/by-sa/4.0/}{Creative Commons Attribution-ShareAlike 4.0 International License}.}


%==============================================================================

\begin{document}

\noindent \makebox[\textwidth]{\textbf{\LARGE Introduction to Rates of Change}}

\quad

%% Update cover image
\begin{figure}[h]
	\centering
	\includegraphics[height=0.4\textwidth]{figures/stables.jpg} % Stables (Stallungen), Franz Marc
\end{figure}

%==============================================================================
\section*{Introduction}

\textcolor{red}{Introductory paragraphs}

%%%
\subsection*{Learning Objectives}

After completing this project, you should be able to:
\begin{itemize}
	\item \textcolor{red}{List of learning objectives}
\end{itemize}

% Data set to use: Weather balloon data. Some of them have sufficiently-interesting latitude or longitude functions. Try finding a series with an interesting longitude function, since longitude can be directly converted into meters easily while latitude depends on the longitude.

% Start with position data and attempting to find velocity at a particular point, which seems like it should be simple.
% Find the average velocity over a given time span by hand, then using Excel.
% Shrink the time span and note the pattern.
% Plot the velocity estimates as a function of all possible average velocities and graph the result, which has a hole in it.
% Now repeat the previous, but explained from a graphical perspective. Plot the position function. Find what the first average velocity corresponds to graphically, in terms of the two position points (it's the slope of the line between them). Work towards the idea that the instantaneous rate of change in the function is the same thing as the slope of its graph.
% There's nothing special about the one time we were looking at earlier, so use "small" step sizes to approximate the rate-of-change function of the position function, i.e. the velocity function of the object. Note the graphical relationship between the two.
% Repeat the same for a different type of data, like temperature as a function of latitude/longitude, to clarify that rates of change work even when we're not talking about position versus time. Graphically estimate the time at which the temperature was increasing the fastest, or when the temperature was (at least momentarily) constant.
% Finally go through a few basic families of function, like sine, exp, and log. Estimate the rate of change function for each, plot against the original, and come up with a conjecture for what its rate of change function is.

% (Overall: Start with a very small data series and/or function, simple enough for the student to work with by hand, then move up to a very large data series.)
% (Overall: The eventual goal is to get students working on open-ended projects and communicating technical information. Rather than making this a blow-by-blow sequence of tasks to complete, present each major section as being its own open-ended investigation where it's up to the student to answer one overarching question.)
% In introducing the relationship between slopes and rates of change, ask the student why converting a numerical question into a geometric one might be useful. There are lots of possible answers, including the fact that it makes things easier to visualize and understand, and that it allows us to use all of the geometric things we know from precalculus (the idea of converting one kind of mathematical question into another is very common and useful since it allows us to reuse previous work to answer new questions).
% Start with a position function, and have the student compute some average velocities between different points.
% Then do the same thing, this time with a table of position values that the student generates from the given function.
% Introduce the idea of instantaneous velocity, and how it relates to slope.
% Estimate instantaneous velocity at a point, then repeat for all points to estimate the velocity function.
	% Answer some questions about how the two relate to each other geometrically.
% Focus on one point, and estimate the instantaneous velocity there using different step sizes. Note that they seem to settle down as the step size approaches zero.
% Plot the average velocity as a function of the step size. Note the #DIV/0! error when the step size is exactly zero, and the corresponding hole in the graph, but also note that everything seems well-behaved around that point, and motivate the idea of using limits to find what point "should" go there.
% Introduce everything in terms of a "rate-of-change" function, which in the special case of position is just "velocity".

% Go through a few common functions like e^x and sin(x) to explore their rates of change, and have the student conjecture what their rate-of-change functions are. Maybe also try sin(2x) and sin(x^2), and have the student explain why the rate-of-change function is shaped the way that it is.

%==============================================================================
\section{\textcolor{red}{Section Title}}
\label{sec:sectiontitle}

\textcolor{red}{Section description}

%%%
\subsection*{Activity}

\textcolor{red}{Activity introduction}

\begin{enumerate}[\bf {\thesection}a.]
	\item \textcolor{red}{Parts of activity}
\end{enumerate}

\end{document}
